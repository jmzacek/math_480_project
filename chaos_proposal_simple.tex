\documentclass[11pt]{article}
%Gummi|065|=)
\title{\textbf{Math 480 - Course Project}
			  \\Creating a schedule for "Chaos Week"}
\author{Derek Rhodeham\\
		Justin Zecak\\
		Davis Zvejnieks}
\date{}
\usepackage{enumitem}
\usepackage{url}
\usepackage{amsmath}

\begin{document}

\maketitle

\section{Description}
Dr. Jonah Ostroff runs MathILy, a 6-week math camp for high school students. Between the the first three weeks and the last two weeks is "chaos week." During "chaos week" a new set of classes are voted on by students, and they are enrolled in a week long program containing their top five classes.\\
\indent In 2015, there were around 30 available topics, which students would vote for the most interesting classes. These classes were narrowed down to 10 classes, which the three instructors could teach. Each student could then vote 0 to 3 on the remaining classes, 0 representing no interest, 3 representing most interest. Each student was then assigned five classes, which they would participate in daily for the week. Students were assigned classes not only by their most preffered class, but also by instructor overrides based on their knowledge and performance of the student. There were 5 time slots available, so 2 classes were taught at the same time.\\
\indent Dr. Ostroff would like a schedule for 2016 so that there are no time conflicts between classes an instructor could teach while maximizing the preferences of students. This year the schedule should accommodate 24 students, 4 to 5 instructors, and 15 classes. This would mean three classes will be taught at each of the 5 time slots.


\section{Impact}
Dr. Ostroff had used a method for creating the schedule by searching for a bipartite graph to represent classes which would share a timeslot. This method does not easily scale to finding a 3-matching in a hypergraph. More complexity arises if the scheduleere to accomodate 3 or more simultaneous classes.\\
\indent This method in 2015 could be solved by hand by representing the schedule of 10 classes as a $K_{10}$ graph. Edges would be removed from the graph, if a time conflict occured with an instructor or student. Then a bipartite graph was found, with each pair of vertices representing classes to occur simultaneous. Additionally, the staff would initially assign about 30\% of the classes before using this method, and only after the schedule was made were the rest of the classes assigned.\\
\indent When scaling this problem to three simultaneous problems, this method becomes more time consuming. We propose a method to automate the process of finding a 3-matching using a computer program to eliminate the time needed to find a working schedule. By translating the problem into a linear programming (LP) problem the schedule would not only be easier to compute, but can be scaled to larger schedules using various LP relaxation methods.


\section{Methods}

We intend to use LP solving algorithms the issue of 3-matching. By finding sets of 3 connected vertices, we can determine which classes can occur in the same time slot. We will examine relaxation methods and their respective accuracies and time complexities. This will be useful for establishing a method to use on for larger schedules if needed for MathILy in the future. In addition, heuristic approaches will also be examined to reduce computation time for larger schedules.


\section{References}
Yuk Hei Chan, a PhD candidate at the University of Maryland, wrote his masters thesis on linear programming relaxations of matching problems in hypergraphs\footnote{\url{https://www.cs.umd.edu/~yhchan/thesis.pdf}}. He discusses in his in his thesis the time complexity of finding k-matchings, integrality gap of various LP relaxations, and methods to improve the LP relaxations. Chan mentions a polynomial time approximation of k-matching, which could be applied for larger data sets. He also compares k-matching to other problems, such as set packing. This resource will be invaluable in searching for ways to improve schedule creation.



\end{document}
